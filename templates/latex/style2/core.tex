% core.tex - foundational template for latex document styling
% provides base configuration for fonts, math, languages, and typography
% designed to be imported by other style templates

% --- CORE PACKAGES ---
% fontspec provides modern font handling for xetex/luatex engines
\usepackage{fontspec}

% --- ENGINE-SPECIFIC CONFIGURATION ---
% conditional loading of packages based on tex engine in use
\usepackage{ifxetex,ifluatex}

% xetex-specific additions
\ifxetex
  % xltxtra provides extra features for xetex
  % xunicode ensures proper unicode handling
  \usepackage{xltxtra,xunicode}
\fi

% luatex-specific additions
\ifluatex
  % extra features and fixes for luatex
  \usepackage{luatextra}
\fi

% --- MATH AND SPECIAL CHARACTER HANDLING ---
\usepackage{amsmath}
\usepackage{amssymb}
\usepackage{amsfonts}
\usepackage{mathtools}

\ifluatex
  \usepackage{unicode-math}
  \usepackage{luacode}
  
  \begin{luacode*}
  function replace_math_chars(s)
    -- Greek lowercase
    s = unicode.utf8.gsub(s, "α", "\\ensuremath{\\alpha}")
    s = unicode.utf8.gsub(s, "β", "\\ensuremath{\\beta}")
    s = unicode.utf8.gsub(s, "γ", "\\ensuremath{\\gamma}")
    s = unicode.utf8.gsub(s, "δ", "\\ensuremath{\\delta}")
    s = unicode.utf8.gsub(s, "ε", "\\ensuremath{\\varepsilon}")
    s = unicode.utf8.gsub(s, "η", "\\ensuremath{\\eta}")
    s = unicode.utf8.gsub(s, "θ", "\\ensuremath{\\theta}")
    s = unicode.utf8.gsub(s, "κ", "\\ensuremath{\\kappa}")
    s = unicode.utf8.gsub(s, "λ", "\\ensuremath{\\lambda}")
    s = unicode.utf8.gsub(s, "μ", "\\ensuremath{\\mu}")
    s = unicode.utf8.gsub(s, "ρ", "\\ensuremath{\\rho}")
    s = unicode.utf8.gsub(s, "σ", "\\ensuremath{\\sigma}")
    s = unicode.utf8.gsub(s, "τ", "\\ensuremath{\\tau}")
    s = unicode.utf8.gsub(s, "φ", "\\ensuremath{\\phi}")
    s = unicode.utf8.gsub(s, "ω", "\\ensuremath{\\omega}")
    
    -- Greek uppercase
    s = unicode.utf8.gsub(s, "Γ", "\\ensuremath{\\Gamma}")
    s = unicode.utf8.gsub(s, "Δ", "\\ensuremath{\\Delta}")
    s = unicode.utf8.gsub(s, "Σ", "\\ensuremath{\\Sigma}")
    
    -- Math symbols
    s = unicode.utf8.gsub(s, "→", "\\ensuremath{\\rightarrow}")
    s = unicode.utf8.gsub(s, "↔", "\\ensuremath{\\leftrightarrow}")
    s = unicode.utf8.gsub(s, "⇌", "\\ensuremath{\\rightleftharpoons}")
    s = unicode.utf8.gsub(s, "∇", "\\ensuremath{\\nabla}")
    s = unicode.utf8.gsub(s, "∝", "\\ensuremath{\\propto}")
    s = unicode.utf8.gsub(s, "⁺", "\\ensuremath{^{+}}")
    s = unicode.utf8.gsub(s, "⁻", "\\ensuremath{^{-}}")
    s = unicode.utf8.gsub(s, "₋", "\\ensuremath{_{-}}")
    s = unicode.utf8.gsub(s, "₊", "\\ensuremath{_{+}}")
    s = unicode.utf8.gsub(s, "ₐ", "\\ensuremath{_{a}}")
    
    -- Special symbols for BasisGrotesqueMonoPro
    s = unicode.utf8.gsub(s, "⊑", "\\ensuremath{\\sqsubseteq}")
    s = unicode.utf8.gsub(s, "⊓", "\\ensuremath{\\sqcap}")
    s = unicode.utf8.gsub(s, "∃", "\\ensuremath{\\exists}")
    
    return s
  end
  \end{luacode*}
  
  % Register the callback to process the input
  \AtBeginDocument{%
    \directlua{
      luatexbase.add_to_callback("process_input_buffer", 
                                 replace_math_chars, 
                                 "replace math chars with LaTeX commands")
    }%
  }
  
  % Set up the math font
  \setmathfont{Latin Modern Math}
\fi

\ifxetex
  \usepackage{unicode-math}
  \setmathfont{Latin Modern Math}
\fi

% --- SYMBOL AND CHARACTER SUPPORT ---
% Symbol packages
\usepackage{bbding}    % Various symbols
\usepackage{pifont}    % Dingbats symbols
\usepackage{textcomp}  % Additional text symbols

% Box drawing character substitutions
\usepackage{newunicodechar}

% Define nicer symbols for box drawing characters
\newunicodechar{─}{\textemdash\kern-0.1em\textemdash}
\newunicodechar{│}{\textbar}
\newunicodechar{┼}{\ding{59}}
\newunicodechar{┘}{\ding{217}}
\newunicodechar{└}{\ding{218}}
\newunicodechar{┐}{\ding{216}}
\newunicodechar{┌}{\ding{219}}

% --- LANGUAGE CONFIGURATION ---
% babel with english provides proper hyphenation and language-specific formatting
\usepackage[english]{babel}

% --- ADVANCED TYPOGRAPHY TOOLS ---
% etoolbox provides programming tools for latex
\usepackage{etoolbox}
% fontenc sets up proper font encoding for output
\usepackage[T1]{fontenc}

% Define universal fallback fonts (good Unicode coverage)
\newfontfamily\primaryfallback{Latin Modern Roman}
\newfontfamily\secondaryfallback{DejaVu Serif}

% Register the fallback system with LuaTeX
\directlua{
  -- efficiently store font IDs to avoid repeated lookups
  local fallback_ids = {}
  local main_font_ids = {}
  
  -- high-performance character substitution
  function substitute_missing_chars(head)
    -- process the node list only once
    for n in node.traverse_id(node.id("glyph"), head) do
      local font_id = n.font
      local char_id = n.char
      
      -- check if the current font has this character
      if fonts.chars[font_id] and not fonts.chars[font_id][char_id] then
        -- get fallback font ID (cached for performance)
        if not fallback_ids[1] then
          fallback_ids[1] = font.id("\string\primaryfallback")
          fallback_ids[2] = font.id("\string\secondaryfallback")
        end
        
        -- try primary fallback first
        if fonts.chars[fallback_ids[1]] and fonts.chars[fallback_ids[1]][char_id] then
          n.font = fallback_ids[1]
        -- then secondary fallback
        elseif fonts.chars[fallback_ids[2]] and fonts.chars[fallback_ids[2]][char_id] then
          n.font = fallback_ids[2]
        end
      end
    end
    return head
  end
  
  -- register callback with high priority
  luatexbase.add_to_callback("pre_linebreak_filter", 
                             substitute_missing_chars, 
                             "universal font fallback system")
}

% --- DOCUMENT STYLE (VIBE) CONFIGURATION ---
% set default style to classic if not specified
$if(vibe)$
\def\vibe{$vibe$}
$else$
\def\vibe{classic}
$endif$

% --- FONT COMBINATIONS FOR DIFFERENT VIBES ---
% each vibe defines a specific combination of serif, monospace, and sans-serif fonts
% for a cohesive visual style

% classic vibe: traditional academic style
\ifdefstring{\vibe}{classic}{%
  % alegreya for main text - elegant serif with good readability
  \setmainfont{Alegreya}
  % iosevka for code - clear monospace with good spacing
  \setmonofont{Iosevka}[Scale=0.8]
  % fira sans for headlines - modern sans-serif
  \setsansfont{Fira Sans}
}{}

% alternate vibe: more contemporary feel
\ifdefstring{\vibe}{alternate}{%
  % ebgaramond for sophisticated main text
  \setmainfont{ebgaramond}
  % inconsolata for clean code display
  \setmonofont{inconsolata}
  % fira sans maintains modern headings
  \setsansfont{Fira Sans}
}{}

% forward vibe: modern and progressive
\ifdefstring{\vibe}{forward}{%
  % austera text for contemporary body text
  \setmainfont{Austera Text}
  % native for modern code presentation
  \setmonofont{Native}[Scale=0.8]
  % messina sans for clean headlines
  \setsansfont{Messina Sans}
}{}

% professional vibe: business-oriented design
\ifdefstring{\vibe}{professional}{%
  % palatino nova for classic professional text
  \setmainfont{Palatino Nova}
  % source code pro for technical precision
  \setmonofont{Source Code Pro}[Scale=0.7]
  % helvetica now for modern business feel
  \setsansfont{Helvetica Now Text}
}{}

% orthographic vibe: emphasis on readability
\ifdefstring{\vibe}{orthographic}{%
  % fs ostro for clear body text
  \setmainfont{FS Ostro}
  % native for consistent code display
  \setmonofont{Native}[Scale=0.8]
  % messina sans for balanced headlines
  \setsansfont{Messina Sans}
}{}

% artistic vibe: creative and elegant
\ifdefstring{\vibe}{artistic}{%
  % adobe jenson pro for classical elegance
  \setmainfont{Adobe Jenson Pro}
  % basis grotesque mono for artistic code
  \setmonofont{Basis Grotesque Mono Pro}[Scale=0.7]
  % messina sans for creative headlines
  \setsansfont{Messina Sans}
  
  % --- DEDICATED MATH FONT FOR ARTISTIC VIBE ---
  % Set up Latin Modern Math for all mathematical content
  % This ensures Greek letters and math symbols display correctly
  % while maintaining Adobe Jenson Pro for regular text
  \setmathfont{Latin Modern Math}
  
  % Configure specific ranges to ensure complete coverage
  \setmathfont[range=\mathup/{latin,Latin,greek,Greek,num,symbols}]{Latin Modern Math}
  \setmathfont[range=\mathbfup/{latin,Latin,greek,Greek}]{Latin Modern Math}
  \setmathfont[range=\mathit/{latin,Latin,greek,Greek}]{Latin Modern Math}
  \setmathfont[range=\mathbfit/{latin,Latin,greek,Greek}]{Latin Modern Math}
}{}

% playful vibe: casual and approachable
\ifdefstring{\vibe}{playful}{%
  % comic neue for friendly body text
  \setmainfont{Comic Neue}
  % courier prime for casual code
  \setmonofont{Courier Prime}[Scale=MatchLowercase]
  % quicksand for fun headlines
  \setsansfont{Quicksand}
}{}

% --- HEADING STYLES CONFIGURATION ---
% consistent but minimal heading formatting
% using rmfamily (serif) and mdseries (medium weight) for elegance
\setkomafont{title}{\rmfamily\mdseries\upshape\normalsize}
\setkomafont{sectioning}{\rmfamily\mdseries\upshape\normalsize}
\setkomafont{descriptionlabel}{\rmfamily\mdseries\upshape\normalsize}
\setkomafont{subsubsection}{\rmfamily\mdseries\upshape\normalsize\underline}

% --- FOOTNOTE FORMATTING ---
% flush left margins for cleaner footnote appearance
\usepackage[flushmargin]{footmisc-2011-06-06}

% --- PARAGRAPH SPACING ---
% remove indentation for modern look
\setlength{\parindent}{0pt}
% set vertical space between paragraphs
% plus/minus values allow tex to adjust spacing when needed
\setlength{\parskip}{6pt plus 2pt minus 1pt}

% --- ENUMERATION SUPPORT ---
% conditional setup for fancy enumeration if requested
$if(fancy-enums)$
\makeatletter\AtBeginDocument{%
  % set consistent label width for lists
  \renewcommand{\@listi}
    {\setlength{\labelwidth}{4em}}
}\makeatother
% enable enhanced enumeration features
\usepackage{enumerate}
$endif$

% --- TABLE SUPPORT ---
$if(tables)$
% array package for advanced table features
\usepackage{array}
% commands to handle backslashes in tables
\newcommand{\PreserveBackslash}[1]{\let\temp=\\#1\let\\=\temp}
\let\PBS=\PreserveBackslash
$endif$

% --- SUBSCRIPT SUPPORT ---
$if(subscript)$
% command for properly formatted subscript text
\newcommand{\textsubscr}[1]{\ensuremath{_{\scriptsize\textrm{#1}}}}
$endif$

% --- HYPERLINK CONFIGURATION ---
% enable hyperlinks with customized appearance
\usepackage[breaklinks=true]{hyperref}
\hypersetup{colorlinks,%
citecolor=blue,%
filecolor=blue,%
linkcolor=blue,%
urlcolor=blue}

% --- URL HANDLING ---
$if(url)$
% enable proper url formatting
\usepackage{url}
$endif$

% --- GRAPHICS SUPPORT ---
$if(graphics)$
\usepackage{graphicx}
% automatically scale images to fit page width
\makeatletter
\def\maxwidth{\ifdim\Gin@nat@width>\linewidth\linewidth
\else\Gin@nat@width\fi}
\makeatother
% redefine includegraphics to use maxwidth by default
\let\Oldincludegraphics\includegraphics
\renewcommand{\includegraphics}[1]{\Oldincludegraphics[width=\maxwidth]{#1}}
$endif$

% --- SECTION NUMBERING ---
% control section numbering depth
$if(numbersections)$
$else$
% disable section numbering if not requested
\setcounter{secnumdepth}{0}
$endif$

% --- VERBATIM IN FOOTNOTES ---
$if(verbatim-in-note)$
% enable proper handling of verbatim text in footnotes
\usepackage{fancyvrb}
\VerbatimFootnotes
$endif$

% --- CUSTOM HEADER INCLUDES ---
% process any additional header code
$for(header-includes)$
$header-includes$
$endfor$

% --- INCLUDE COMMON TEMPLATE ---
% load shared template configurations
$common.tex()$