% core.tex - foundational template for latex document styling
% provides base configuration for fonts, math, languages, and typography
% designed to be imported by other style templates

% --- core font and language packages ---
% fontspec provides modern font handling for xetex/luatex engines
\usepackage{fontspec}

% --- engine-specific configuration ---
% conditional loading of packages based on tex engine in use
\usepackage{ifxetex,ifluatex}

% xetex-specific additions
\ifxetex
  % xltxtra provides extra features for xetex
  % xunicode ensures proper unicode handling
  \usepackage{xltxtra,xunicode}
\fi

% luatex-specific additions
\ifluatex
  % extra features and fixes for luatex
  \usepackage{luatextra}
\fi

% --- mathematical support ---
% amsmath for advanced math formatting and environments
\usepackage{amsmath}
% amssymb for additional mathematical symbols
\usepackage{amssymb}
% amsfonts for specialized math fonts
\usepackage{amsfonts}

% --- language configuration ---
% babel with english provides proper hyphenation and language-specific formatting
\usepackage[english]{babel}

% --- advanced typography tools ---
% etoolbox provides programming tools for latex
\usepackage{etoolbox}
% fontenc sets up proper font encoding for output
\usepackage[T1]{fontenc}

% --- document style (vibe) configuration ---
% set default style to classic if not specified
$if(vibe)$
\def\vibe{$vibe$}
$else$
\def\vibe{classic}
$endif$

% --- font combinations for different vibes ---
% each vibe defines a specific combination of serif, monospace, and sans-serif fonts
% for a cohesive visual style

% classic vibe: traditional academic style
\ifdefstring{\vibe}{classic}{%
  % alegreya for main text - elegant serif with good readability
  \setmainfont{Alegreya}
  % iosevka for code - clear monospace with good spacing
  \setmonofont{Iosevka}[Scale=0.8]
  % fira sans for headlines - modern sans-serif
  \setsansfont{Fira Sans}
}{}

% alternate vibe: more contemporary feel
\ifdefstring{\vibe}{alternate}{%
  % ebgaramond for sophisticated main text
  \setmainfont{ebgaramond}
  % inconsolata for clean code display
  \setmonofont{inconsolata}
  % fira sans maintains modern headings
  \setsansfont{Fira Sans}
}{}

% forward vibe: modern and progressive
\ifdefstring{\vibe}{forward}{%
  % austera text for contemporary body text
  \setmainfont{Austera Text}
  % native for modern code presentation
  \setmonofont{Native}[Scale=0.8]
  % messina sans for clean headlines
  \setsansfont{Messina Sans}
}{}

% professional vibe: business-oriented design
\ifdefstring{\vibe}{professional}{%
  % palatino nova for classic professional text
  \setmainfont{Palatino Nova}
  % source code pro for technical precision
  \setmonofont{Source Code Pro}[Scale=0.7]
  % helvetica now for modern business feel
  \setsansfont{Helvetica Now Text}
}{}

% orthographic vibe: emphasis on readability
\ifdefstring{\vibe}{orthographic}{%
  % fs ostro for clear body text
  \setmainfont{FS Ostro}
  % native for consistent code display
  \setmonofont{Native}[Scale=0.8]
  % messina sans for balanced headlines
  \setsansfont{Messina Sans}
}{}

% artistic vibe: creative and elegant
\ifdefstring{\vibe}{artistic}{%
  % adobe jenson pro for classical elegance
  \setmainfont{Adobe Jenson Pro}
  % basis grotesque mono for artistic code
  \setmonofont{Basis Grotesque Mono Pro}[Scale=0.7]
  % messina sans for creative headlines
  \setsansfont{Messina Sans}
}{}

% playful vibe: casual and approachable
\ifdefstring{\vibe}{playful}{%
  % comic neue for friendly body text
  \setmainfont{Comic Neue}
  % courier prime for casual code
  \setmonofont{Courier Prime}[Scale=MatchLowercase]
  % quicksand for fun headlines
  \setsansfont{Quicksand}
}{}

% --- heading styles configuration ---
% consistent but minimal heading formatting
% using rmfamily (serif) and mdseries (medium weight) for elegance
\setkomafont{title}{\rmfamily\mdseries\upshape\normalsize}
\setkomafont{sectioning}{\rmfamily\mdseries\upshape\normalsize}
\setkomafont{descriptionlabel}{\rmfamily\mdseries\upshape\normalsize}
\setkomafont{subsubsection}{\rmfamily\mdseries\upshape\normalsize\underline}

% --- footnote formatting ---
% flush left margins for cleaner footnote appearance
\usepackage[flushmargin]{footmisc-2011-06-06}

% --- paragraph spacing ---
% remove indentation for modern look
\setlength{\parindent}{0pt}
% set vertical space between paragraphs
% plus/minus values allow tex to adjust spacing when needed
\setlength{\parskip}{6pt plus 2pt minus 1pt}

% --- enumeration support ---
% conditional setup for fancy enumeration if requested
$if(fancy-enums)$
\makeatletter\AtBeginDocument{%
  % set consistent label width for lists
  \renewcommand{\@listi}
    {\setlength{\labelwidth}{4em}}
}\makeatother
% enable enhanced enumeration features
\usepackage{enumerate}
$endif$

% --- table support ---
$if(tables)$
% array package for advanced table features
\usepackage{array}
% commands to handle backslashes in tables
\newcommand{\PreserveBackslash}[1]{\let\temp=\\#1\let\\=\temp}
\let\PBS=\PreserveBackslash
$endif$

% --- subscript support ---
$if(subscript)$
% command for properly formatted subscript text
\newcommand{\textsubscr}[1]{\ensuremath{_{\scriptsize\textrm{#1}}}}
$endif$

% --- hyperlink configuration ---
% enable hyperlinks with customized appearance
\usepackage[breaklinks=true]{hyperref}
\hypersetup{colorlinks,%
citecolor=blue,%
filecolor=blue,%
linkcolor=blue,%
urlcolor=blue}

% --- url handling ---
$if(url)$
% enable proper url formatting
\usepackage{url}
$endif$

% --- graphics support ---
$if(graphics)$
\usepackage{graphicx}
% automatically scale images to fit page width
\makeatletter
\def\maxwidth{\ifdim\Gin@nat@width>\linewidth\linewidth
\else\Gin@nat@width\fi}
\makeatother
% redefine includegraphics to use maxwidth by default
\let\Oldincludegraphics\includegraphics
\renewcommand{\includegraphics}[1]{\Oldincludegraphics[width=\maxwidth]{#1}}
$endif$

% --- section numbering ---
% control section numbering depth
$if(numbersections)$
$else$
% disable section numbering if not requested
\setcounter{secnumdepth}{0}
$endif$

% --- verbatim in footnotes ---
$if(verbatim-in-note)$
% enable proper handling of verbatim text in footnotes
\usepackage{fancyvrb}
\VerbatimFootnotes
$endif$

% --- custom header includes ---
% process any additional header code
$for(header-includes)$
$header-includes$
$endfor$

% --- include common template ---
% load shared template configurations
$common.tex()$